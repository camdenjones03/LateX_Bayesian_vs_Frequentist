\documentclass[article]{beamer}
\usetheme{Madrid} %minimalist theme that doesn't distract from slides
\usecolortheme{seahorse} %light shade of blue
\setbeamertemplate{footline}[frame number]

%Add packages if needed
\usefonttheme[]{serif}
\usepackage{soul}
\usepackage{amsmath, latexsym, color, graphicx, amssymb, bm, here}
\usepackage{epsf, epsfig, pifont,tikz,subfigure}
\usepackage{graphics, calrsfs}
\usepackage{times}
\usepackage{fancybox,calc}
\usepackage{palatino,mathpazo}
\usepackage{amsfonts}
\usepackage{sidecap}
\usepackage{multirow}
\usepackage{multicol}
%--------------------------------------------------
% Title, Author, Email
%--------------------------------------------------
\title{Bayesian Statistics}
\author{Camden Jones \newline {\footnotesize}}
\institute{Middle Georgia State University \newline Email: camden.jones1@mga.edu \vspace{.2cm}}
\date{\scriptsize{Math 3207 Final Presentation}}


\AtBeginSection[]
{
%  \begin{frame}
  %\tableofcontents[currentsection]
% \end{frame}
 }
  


\begin{document}

\maketitle

%\begin{frame}           % Frames are Slides in Latex
%\frametitle{Contents}   % Include contents when you have a few sections
%\tableofcontents
%\end{frame}

\section{Introduction}  % First section of slides

\begin{frame}{fragile}
\frametitle{Statistics and Probability}
\begin{itemize}
    \item What is \textbf{Statistics}?
    \vspace{1cm}
    \begin{enumerate}
        \item Parameters
        \vspace{0.5cm}
        \item Hypotheses
        \vspace{0.5cm}
        \item Probability
    \end{enumerate}
    \vspace{1cm}
    \item Two major approaches: \textbf{\textcolor{blue}{Frequentist}} and \textbf{\textcolor{green}{Bayesian}}
\end{itemize}    
\end{frame}


\begin{frame}[fragile]
\frametitle{Frequentist vs. Bayesian} %Affects only title
\resizebox{\textwidth}{!}{%
    \begin{tabular}{|c|c|}
    \hline
       \textbf{\textcolor{blue}{Frequentist}}  & \textbf{\textcolor{green}{Bayesian}} \\ \hline
       $ \text{Hypothesis} \rightarrow \text{Data Observed}$ & $\text{Data Observed} \rightarrow \text{Hypothesis}$ \\
       \hline
       Parameters are fixed & Parameters are random variables\\ \hline
       No belief or uncertainties & Update beliefs with new data \\ \hline
    \end{tabular} }
    \label{tab:my_label}
\includegraphics[height=3.5cm]{normal.png} \hspace{.2 cm}
\includegraphics[height=3cm]{joker.png}
%\hspace{0.4cm}  \textbf{Edge Coloring}: $\chi'(G)=2$ \hspace{0.8cm} \textbf{List-Edge Coloring}: $\chi'_\ell(G)=2$
%\begin{figure}
%\centering
%\includegraphics[height=8cm]{list.png}
%\end{figure}

\end{frame}


\begin{frame}[fragile]
\frametitle{History}

\begin{block}{Discovery}  % Use block for important theorems
Reverend Thomas Bayes - An Essay Towards Solving a Problem in the Doctrine of Chances
\end{block}

\begin{block}{Proliferation}
Found by Bayes' friend Richard Price after his death, who had it published in 1763.
\end{block}
\vspace{.4cm}

%\begin{center}
%\def\arraystretch{1.5}
%\begin{tabular}{c|c|c}
 %    \hline
  %   Author & Given & Implication  \\
   %  \hline
    % Galvin & Bipartite & $\chi'(G)=\chi'_\ell(G)$\\
     %Cohen \& Havet & Planar \& $\Delta\geq 9$ & $\chi'_\ell(G)\leq\Delta+1$\\
%\end{tabular}
%\end{center}

\end{frame}

\begin{frame}{Important Definitions}

\begin{enumerate}
    \item Conditional Probability
    \begin{itemize}
        \item $P(A|B)$
    \end{itemize}
    \item Probability Distribution \\
    \begin{tabular}{|c|c|c|c|c|c|c|}
    \hline
       $X$  & 1 & 2 & 3 & 4 & 5 & 6   \\
       \hline
       $P(X)$ &  $\frac{1}{6}$ & $\frac{1}{6}$ & $\frac{1}{6}$ & $\frac{1}{6}$ & $\frac{1}{6}$  & $\frac{1}{6}$ \\
       \hline
    \end{tabular}
    \item Prior Distribution
    \begin{itemize}
        \item $P(A)$
    \end{itemize}
    \item Posterior Distribution
    \begin{itemize}
        \item $P(A|B)$
    \end{itemize}
    \item Likelihood Function
    \begin{itemize}
        \item $P(B|A)$
    \end{itemize}
\end{enumerate}
\end{frame}
\section{Theorem}  % New Section
\begin{frame}{Conditional Probability}
\begin{center}
$P(A|B)=\dfrac{P(A\cap B)}{P(B)},$ if $P(B)\neq 0$ \\
where $P(A\cap B)$ is the probability of both A and B. \\
\vspace{1cm}
$P(B|A)=\dfrac{P(A\cap B)}{P(A)}$, if $P(A)\neq 0$ \\
\vspace{1cm}
    $P(A\cap B)= P(B|A)P(A)$ \\
\vspace{1cm}
    \large $P(A|B)=\dfrac{P(B|A)P(A)}{P(B)}$, if $P(B)\neq 0$.
\end{center}
\end{frame}
\begin{frame}{Theorem}

\begin{block}{Bayes' Theorem}
$P(A|B) = \frac{P(B|A)P(A)}{P(B)}$
\end{block}

\end{frame}

\begin{frame}{An Example}
\begin{center}
\includegraphics[height=3cm]{first.png} 
\includegraphics[height=3cm]{left.png} \\
$P(R_1)=$ likelihood the 1st ball lands on the right   \\
$P(R_2)=$ likelihood the 2nd ball lands to the right of the 1st\\
Assumptions: $P(R_1)=0.5$, $P(R_2)=0.5$, $P(R_2|R_1)=0.25$\\
\vspace{1 cm}

 $P(R_1|R_2)=\dfrac{P(R_2|R_1)P(R_1)}{P(R_2)} =\dfrac{0.25(0.5)}{0.5}=0.25$ 
\end{center}
\end{frame}

\begin{frame}{Prominent Uses}
\begin{center}
\begin{itemize}
    \item Finance: Predicting lending risks (Bijak et al.).
    \item Medical Research: New way to analyze drug trials (Wijeysundera et al.).
    \item Artificial Intelligence: Improving search engines by eliminating bias (Yang et al.).
\end{itemize}
\end{center}

\end{frame}

\section{Conclusion}


\begin{frame}{}
   \begin{center}
     \Huge{Thank You!}
   \end{center}
\end{frame}
\end{document}
